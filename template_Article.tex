\newgeometry{left=5pt,right=10pt,top=5pt,bottom=0pt}
\begin{multicols*}{2}
\fontsize{13pt}{0}
\begin{tblr}{colspec={|Q[c,8mm]|Q[c,8mm]|Q[c,8mm]|Q[c,8mm]|Q[c,8mm]|Q[c,8mm]|Q[c,8mm]|Q[c,8mm]|},
		rowspec={|Q[c, 8mm]|Q[c, 8mm]|Q[c, 8mm]|Q[c, 8mm]|Q[c, 8mm]|Q[c, 8mm]|Q[c, 8mm]|Q[c, 8mm]|}}
	$2^0$ & $2^1$ & \SetCell{bg=azure7} $2^2$ & \SetCell{bg=azure7} $2^3$ & $2^4$ & \SetCell{bg=azure7} $2^5$ & $2^6$ & \SetCell{bg=azure7} $2^7$ \\ 
	$2^{8}$ & $2^{9}$ & $2^{10}$ & $2^{11}$ & $2^{12}$ & \SetCell{bg=azure7} $2^{13}$ & $2^{14}$ & $2^{15}$ \\ 
	$2^{16}$ & \SetCell{bg=azure7} $2^{17}$ & $2^{18}$ & \SetCell{bg=azure7} $2^{19}$ & $2^{20}$ & $2^{21}$ & $2^{22}$ & $2^{23}$ \\ 
	$2^{24}$ & $2^{25}$ & $2^{26}$ & $2^{27}$ & $2^{28}$ & $2^{{29}}$ & $2^{30}$ & \SetCell{bg=azure7} $2^{31}$ \\ 
	$2^{32}$ & $2^{33}$ & $2^{34}$ & $2^{35}$ & $2^{36}$ & $2^{37}$ & $2^{38}$ & $2^{39}$ \\ 
	$2^{40}$ & $2^{41}$ & $2^{42}$ & $2^{43}$ & $2^{44}$ & $2^{45}$ & $2^{46}$ & $2^{47}$ \\ 
	$2^{48}$ & $2^{49}$ & $2^{50}$ & $2^{51}$ & $2^{52}$ & $2^{53}$ & $2^{54}$ & $2^{55}$ \\ 
	$2^{56}$ & $2^{57}$ & $2^{58}$ & $2^{59}$ & $2^{60}$ & \SetCell{bg=azure7} $2^{61}$ & $2^{62}$ & $2^{63}$ \\ 
\end{tblr}
\vspace{3mm}
\break
\small{Рис. 1.}
\newline
\newline
\setlength{\parskip}{0cm}
число зерен в формуле Евклида определяется выражением $2^n - 1$. \space Если это число простое, то, умножив его на число в предыдущей клетке, то есть на $2^{n-1}$, получим совершенное число. (см. рис. 1).\par
  Простые числа ряда $2^n - 1$ называют числами Мерсена по имени французского математика XVII века, занимавшегося их изучением *). \space \space На
\noindent\rule{10cm}{0.4pt}
\newline
{\small{*) См. также <<Квант>>, 1971, №8, с. 3.}}
\includegraphics[width=0.99\linewidth]{./Screenshot_1.png}
72
\newline
рисунке 1 закрашены те клетки, в которых после вычитания 1 получаются числа Мерсенна. Таких клеток на доске 9 - им соответствуют первые девять совершенных чисел.\par
Совершенные числа обладают рядом таинственных и вместе с тем замечательных свойств. Например, все совершенные числа<<треугольные>>.Это означает, что если взять, допустим, шарики в количестве, равном совершенному числу, то их можно расположить так, что они образуют равносторонний треугольник.\par
Иначе говоря,  каждое совершенное число есть сумма вида $1 + 2 + 3 + 4 + ... + n$\par
Также легко можно заметить, что каждое совершенное число, за исключением 6, есть частичная сумма ряда из кубов нечетных чисел $1^3 + 3^3 + 5^3 + ...$\par
А вот еще одно свойство совершенных чисел: сумма обратных значений делителей совершенного числа, включая и само число как делитель, всегда равны 2. Так,для числа 28 имеем\\\par
$\frac{1}{1} + \frac{1}{2} + \frac{1}{4} + \frac{1}{7} + \frac{1}{14} + \frac{1}{28} = 2.$\\\par
До сегодняшнего дня остаются без ответа два важных вопроса: существует ли нечетное совершенное число?
До сих пор не найдено ни одного нечетного совершенного числа, но вместе с тем и не доказано, что такого числа не существует. Ответ на второй вопрос зависит от того, является ли ряд простых чисел Мерсенна бесконечным, так как каждое простое число этого ряда приводит к совершенному числу. Было замечено, что при подстановке первых четырех чисел Мерсенна (3, 7, 31, 127) вместо n в формулу $2^n - 1$ снова получаются числа Марсенна.\\
\begin{center}
	\fontsize{19pt}{0}
$F = \pi\upsilon^2\frac{\rho\rho_0}{\rho - \rho_0}r^2$\\
\end{center}
и\\
\begin{center}
	\fontsize{19pt}{0}
$\upsilon = \sqrt{\frac{F}{\pi r^2} \frac{\rho - \rho_0}{\rho\rho_0}}$
\end{center}
\begin{flushright}
	И.Ш. \textit{Слабодецкий}
\end{flushright}
\end{multicols*}
